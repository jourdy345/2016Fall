% \title{Economic Time Series HW2}
% \author{Daeyoung Lim}

% \documentclass[answers]{exam}
\documentclass{article}
\usepackage[left=3cm,right=3cm,top=3.5cm,bottom=2cm]{geometry}
\usepackage{amssymb,amsmath,amsfonts,amsthm}
\usepackage{mathtools}
\usepackage{graphicx}
\usepackage{kotex}
\usepackage[utf8]{inputenc}
\usepackage[T1]{fontenc}
\usepackage{lmodern}
% \usepackage{enumerate}
\usepackage{listings}
\usepackage{courier}
\usepackage{cancel}
\usepackage{array}
\usepackage{courier}
\usepackage{booktabs}
\usepackage{titlesec}
\usepackage[shortlabels]{enumitem}
\usepackage{setspace}
\usepackage{newtxtext}
\usepackage[lite,nofontinfo,zswash,straightbraces]{mtpro2}
\usepackage{empheq}
\usepackage{tikz}
\usepackage{listings}

% \usepackage[toc,page]{appendix}

\setlength{\heavyrulewidth}{1.5pt}
\setlength{\abovetopsep}{4pt}

\DeclarePairedDelimiter{\ceil}{\lceil}{\rceil}
\newcommand\encircle[1]{%
  \tikz[baseline=(X.base)] 
    \node (X) [draw, shape=circle, inner sep=0] {\strut #1};}
 
% Command "alignedbox{}{}" for a box within an align environment
% Source: http://www.latex-community.org/forum/viewtopic.php?f=46&t=8144
\newlength\dlf  % Define a new measure, dlf
\newcommand\alignedbox[2]{
% Argument #1 = before & if there were no box (lhs)
% Argument #2 = after & if there were no box (rhs)
&  % Alignment sign of the line
{
\settowidth\dlf{$\displaystyle #1$}  
    % The width of \dlf is the width of the lhs, with a displaystyle font
\addtolength\dlf{\fboxsep+\fboxrule}  
    % Add to it the distance to the box, and the width of the line of the box     ㅊ
\hspace{-\dlf}  
    % Move everything dlf units to the left, so that & #1 #2 is aligned under #1 & #2
\boxed{#1 #2}
    % Put a box around lhs and rhs
}
}
\setcounter{secnumdepth}{4}
\lstset{
         basicstyle=\footnotesize\ttfamily, % Standardschrift
         %numbers=left,               % Ort der Zeilennummern
         numberstyle=\tiny,          % Stil der Zeilennummern
         %stepnumber=2,               % Abstand zwischen den Zeilennummern
         numbersep=5pt,              % Abstand der Nummern zum Text
         tabsize=2,                  % Groesse von Tabs
         extendedchars=true,         %
         breaklines=true,            % Zeilen werden Umgebrochen
         keywordstyle=\color{red},
            frame=b,         
 %        keywordstyle=[1]\textbf,    % Stil der Keywords
 %        keywordstyle=[2]\textbf,    %
 %        keywordstyle=[3]\textbf,    %
 %        keywordstyle=[4]\textbf,   \sqrt{\sqrt{}} %
         stringstyle=\color{white}\ttfamily, % Farbe der String
         showspaces=false,           % Leerzeichen anzeigen ?
         showtabs=false,             % Tabs anzeigen ?
         xleftmargin=17pt,
         framexleftmargin=17pt,
         framexrightmargin=5pt,
         framexbottommargin=4pt,
         %backgroundcolor=\color{lightgray},
         showstringspaces=false      % Leerzeichen in Strings anzeigen ?        
 }
 \lstloadlanguages{% Check Dokumentation for further languages ...
         %[Visual]Basic
         %Pascal
         %C
         %C++
         %XML
         %HTML
         Java
 }
    %\DeclareCaptionFont{blue}{\color{blue}} 

\definecolor{myblue}{RGB}{72, 165, 226}
\definecolor{myorange}{RGB}{222, 141, 8}
\titleformat{\paragraph}
{\normalfont\normalsize\bfseries}{\theparagraph}{1em}{}
\titlespacing*{\paragraph}
{0pt}{3.25ex plus 1ex minus .2ex}{1.5ex plus .2ex}
\setlength{\heavyrulewidth}{1.5pt}
\setlength{\abovetopsep}{4pt}
\setlength{\parindent}{0mm}
\linespread{1.3}
\DeclareMathOperator{\sech}{sech}
\DeclareMathOperator{\csch}{csch}
\DeclareMathOperator*{\argmin}{\arg\!\min}
\DeclareMathOperator{\tr}{tr}
\DeclareMathOperator{\ve}{vec}
% \DeclareMathOperator{\Pro}{P}

\newcommand{\bs}{\boldsymbol}
\newcommand{\opn}{\operatorname}
% \newcommand{\vecc}{\operatorname{vec}}
%%%%%%%%%%%%%%%%%%%%%%%%%%%%%%%%%%%%%%%%%%%%%%%%%%%%%%%
% % We use newtheorem to define theorem-like structures
% %
% % Here are some common ones. . .
%%%%%%%%%%%%%%%%%%%%%%%%%%%%%%%%%%%%%%%%%%%%%%%%%%%%%%%
% \newtheorem{theorem}{Theorem}
% \newtheorem{lemma}{Lemma}
% \newtheorem{proposition}{Proposition}
% \newtheorem{scolium}{Scolium}   %% And a not so common one.
% \newtheorem{definition}{Definition}
% \newenvironment{proof}{{\sc Proof:}}{~\hfill QED}
% \newenvironment{AMS}{}{}
% \newenvironment{keywords}{}{}
%%%%%%%%%%%%%%%%%%%%%%%%%%%%%%%%%%%%%%%%%%%%%%%%%%%%%%%
% %   The first thanks indicates your affiliation
% %
% %  Just the name here.
% %
% % Your mailing address goes at the end.
% %
% % \thanks is also how you indicate grant support
% %
%%%%%%%%%%%%%%%%%%%%%%%%%%%%%%%%%%%%%%%%%%%%%%%%%%%%%%%


\begin{document}
\setstretch{1.5} %줄간격 조정
\newpage
Let the model be
\begin{align}
  B &= (c,A_{1},\ldots,A_{p});\quad (m\times (mp+1))\\
  x_{t} &= \begin{bmatrix}1 \\ y_{t-1}\\ \vdots \\ y_{t-p}\end{bmatrix}; \quad ((mp+1)\times 1)\\
\end{align}
and define
\begin{align}
  X &= \begin{bmatrix} x_{1}'\\ x_{2}'\\ \vdots \\ x_{T}'\end{bmatrix}; \quad (T\times (mp+1))\\
  y &= \begin{bmatrix} (y_{11}\ldots y_{1T})'\\ \vdots \\ (y_{m1}\ldots y_{mT})'\end{bmatrix}; \quad (Tm\times 1)
\end{align}
and $\alpha = \ve(B)$, that is, $m(mp+1)\times 1$ vector. VAR(p) can be written as
\begin{equation}
  y = (I_{m}\otimes X)\alpha + \epsilon,
\end{equation}
where $\epsilon\sim \mathcal{N}(0,\Omega^{-1}\otimes I_{T})$.
\begin{itemize}
  \item $L(\Omega,\alpha\,|\,y,X)\propto |\Omega|^{-T/2}\exp\left[-\dfrac{1}{2}(y-(I_{m}\otimes X)\alpha)'(\Omega^{-1}\otimes I_{T})(y-(I_{m}\otimes X)\alpha)\right] $
  \item $\pi(\alpha)\propto \exp\left[-\dfrac{1}{2}(\alpha-\mu_{\alpha})'\Sigma_{\alpha}^{-1}(\alpha-\mu_{\alpha})\right]$
  \item $\pi(\Omega)\propto |\Omega|^{-(b-m-1)/2}\exp\left[-\dfrac{1}{2}\tr(D^{-1}\Omega^{-1})\right] $
\end{itemize}
\subsection{Estimating $\alpha$}
\begin{align}
  \pi(\alpha\,|\,\text{rest}) &\propto \exp\left[-\dfrac{1}{2}\left(\alpha'\left(\left(I_{m}\otimes X\right)'\left(\Omega^{-1}\otimes I_{T}\right)\left(I_{m}\otimes X\right)+\Sigma_{\alpha}^{-1}\right)\alpha -2\alpha'\left(\Sigma_{\alpha}^{-1}\mu_{\alpha}+\left(I_{m}\otimes X\right)'\left(\Omega^{-1}\otimes I_{m}\right)y\right)\right)\right]
\end{align}
Thus, $\alpha\,|\,\text{rest}\sim\mathcal{N}\left(\mu_{\alpha}^{T},\Sigma_{\alpha}^{T}\right)$ where
\begin{align}
  \Sigma_{\alpha}^{T} &= \left(\left(\Omega^{-1}\otimes X'X\right)+\Sigma_{\alpha}^{-1}\right)^{-1}\\
  \mu_{\alpha}^{T} &= \Sigma_{\alpha}^{p}\left(\Sigma_{\alpha}^{-1}\mu_{\alpha}+\left(\Omega^{-1}\otimes X'\right)y\right)
\end{align}
\subsection{Estimating $\Omega$}
If we define
\begin{equation}
  Y = \begin{bmatrix} y_{11} & y_{21} & \cdots & y_{m1}\\ y_{12} & y_{22}& \cdots & y_{m2} \\ \cdots & \cdots & \ddots & \vdots \\ y_{1T} & y_{2T} & \cdots & y_{mT}\end{bmatrix}; \quad (T\times m)
\end{equation}
In fact, $ y = \ve Y$. By the identity
\begin{align}
  \tr(A_{1}A_{2}A_{3}) &= \ve (A_{1})'(A_{3}\otimes I)\ve (A_{2})\\
  &= \ve (A_{2})'(A_{1}\otimes I)\ve (A_{3})\\
  &= \ve(A_{3})'(A_{2}\otimes I)\ve(A_{1})
\end{align}
we can get the following relation.
\begin{equation}
  \left(y-\left(I_{m}\otimes X\right)\alpha\right)'\left(\Omega^{-1}\otimes I_{T}\right)\left(y-\left(I_{m}\otimes X\right)\alpha\right) = \tr\left(\left(Y'-BX'\right)\left(Y-XB'\right)\Omega^{-1}\right)
\end{equation}
\begin{align}
  \pi(\Omega\,|\,\text{rest}) &\propto |\Omega|^{-(T+b-m-1)/2}\exp\left[-\dfrac{1}{2}\left(\tr\left(D^{-1}\Omega^{-1}\right)+\tr\left((Y'-BX')(Y-XB')\Omega^{-1}\right)\right)\right]\\
  &= |\Omega|^{-(T+b-m-1)/2}\exp\left[-\dfrac{1}{2}\tr\left(\left(D^{-1}+\left(Y'-BX'\right)\left(Y-XB'\right) \right)\Omega^{-1}\right)\right]
\end{align}
Therefore, $\Omega\,|\,\text{rest} \sim \mathcal{W}_{m}\left(T+b,R_{T}\right)$ where
\begin{equation}
 R_{T} = \left(D^{-1}+\left(Y'-BX'\right)\left(Y-XB'\right) \right)^{-1}
\end{equation}
% \firstpageheader{}{}{\bf\large Daeyoung Lim \\ Economic Time Series \\ Fall Semester, 2016}
% \runningheader{Daeyoung Lim}{Economic Time Series}{Fall Semester, 2016}
% \begin{questions}
%   \question
%   Download CD.txt, GDP.txt, CPI.txt
%   \begin{solution}
%     \begin{align}
%         L(\boldsymbol{\Psi},\Omega\,|\,y_{1:T}) = p(y_{1}\,|\,\boldsymbol{\Psi},\Omega)\prod_{t=2}^{T}p(y_{t}\,|\,\boldsymbol{\Psi},\Omega)
%     \end{align}
%   \end{solution}
%   \question
%   Test the hypothesis of white noise for each of the three variables, 
% \end{questions}
\end{document}
