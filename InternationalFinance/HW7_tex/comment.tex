% \title{Economic Time Series HW2}
% \author{Daeyoung Lim}

% \documentclass[answers]{exam}
\documentclass{article}
\usepackage[left=3cm,right=3cm,top=3.5cm,bottom=2cm]{geometry}
\usepackage{amssymb,amsmath,amsfonts,amsthm}
\usepackage{mathtools}
\usepackage{graphicx}
\usepackage{kotex}
\usepackage[utf8]{inputenc}
\usepackage[T1]{fontenc}
\usepackage{lmodern}
% \usepackage{enumerate}
\usepackage{listings}
\usepackage{courier}
\usepackage{cancel}
\usepackage{array}
\usepackage{courier}
\usepackage{booktabs}
\usepackage{titlesec}
\usepackage[shortlabels]{enumitem}
\usepackage{setspace}
\usepackage{newtxtext}
\usepackage[lite,nofontinfo,zswash,straightbraces]{mtpro2}
\usepackage{empheq}
\usepackage{tikz}
\usepackage{listings}

% \usepackage[toc,page]{appendix}

\setlength{\heavyrulewidth}{1.5pt}
\setlength{\abovetopsep}{4pt}

\DeclarePairedDelimiter{\ceil}{\lceil}{\rceil}
\newcommand\encircle[1]{%
  \tikz[baseline=(X.base)] 
    \node (X) [draw, shape=circle, inner sep=0] {\strut #1};}
 
% Command "alignedbox{}{}" for a box within an align environment
% Source: http://www.latex-community.org/forum/viewtopic.php?f=46&t=8144
\newlength\dlf  % Define a new measure, dlf
\newcommand\alignedbox[2]{
% Argument #1 = before & if there were no box (lhs)
% Argument #2 = after & if there were no box (rhs)
&  % Alignment sign of the line
{
\settowidth\dlf{$\displaystyle #1$}  
    % The width of \dlf is the width of the lhs, with a displaystyle font
\addtolength\dlf{\fboxsep+\fboxrule}  
    % Add to it the distance to the box, and the width of the line of the box     ㅊ
\hspace{-\dlf}  
    % Move everything dlf units to the left, so that & #1 #2 is aligned under #1 & #2
\boxed{#1 #2}
    % Put a box around lhs and rhs
}
}
\setcounter{secnumdepth}{4}
\lstset{
         basicstyle=\footnotesize\ttfamily, % Standardschrift
         %numbers=left,               % Ort der Zeilennummern
         numberstyle=\tiny,          % Stil der Zeilennummern
         %stepnumber=2,               % Abstand zwischen den Zeilennummern
         numbersep=5pt,              % Abstand der Nummern zum Text
         tabsize=2,                  % Groesse von Tabs
         extendedchars=true,         %
         breaklines=true,            % Zeilen werden Umgebrochen
         keywordstyle=\color{red},
            frame=b,         
 %        keywordstyle=[1]\textbf,    % Stil der Keywords
 %        keywordstyle=[2]\textbf,    %
 %        keywordstyle=[3]\textbf,    %
 %        keywordstyle=[4]\textbf,   \sqrt{\sqrt{}} %
         stringstyle=\color{white}\ttfamily, % Farbe der String
         showspaces=false,           % Leerzeichen anzeigen ?
         showtabs=false,             % Tabs anzeigen ?
         xleftmargin=17pt,
         framexleftmargin=17pt,
         framexrightmargin=5pt,
         framexbottommargin=4pt,
         %backgroundcolor=\color{lightgray},
         showstringspaces=false      % Leerzeichen in Strings anzeigen ?        
 }
 \lstloadlanguages{% Check Dokumentation for further languages ...
         %[Visual]Basic
         %Pascal
         %C
         %C++
         %XML
         %HTML
         Java
 }
    %\DeclareCaptionFont{blue}{\color{blue}} 

\definecolor{myblue}{RGB}{72, 165, 226}
\definecolor{myorange}{RGB}{222, 141, 8}
\titleformat{\paragraph}
{\normalfont\normalsize\bfseries}{\theparagraph}{1em}{}
\titlespacing*{\paragraph}
{0pt}{3.25ex plus 1ex minus .2ex}{1.5ex plus .2ex}
\setlength{\heavyrulewidth}{1.5pt}
\setlength{\abovetopsep}{4pt}
\setlength{\parindent}{0mm}
\linespread{1.3}
\DeclareMathOperator{\sech}{sech}
\DeclareMathOperator{\csch}{csch}
\DeclareMathOperator*{\argmin}{\arg\!\min}
\DeclareMathOperator{\tr}{tr}
\DeclareMathOperator{\ve}{vec}
% \DeclareMathOperator{\Pro}{P}

\newcommand{\bs}{\boldsymbol}
\newcommand{\opn}{\operatorname}
% \newcommand{\vecc}{\operatorname{vec}}
%%%%%%%%%%%%%%%%%%%%%%%%%%%%%%%%%%%%%%%%%%%%%%%%%%%%%%%
% % We use newtheorem to define theorem-like structures
% %
% % Here are some common ones. . .
%%%%%%%%%%%%%%%%%%%%%%%%%%%%%%%%%%%%%%%%%%%%%%%%%%%%%%%
% \newtheorem{theorem}{Theorem}
% \newtheorem{lemma}{Lemma}
% \newtheorem{proposition}{Proposition}
% \newtheorem{scolium}{Scolium}   %% And a not so common one.
% \newtheorem{definition}{Definition}
% \newenvironment{proof}{{\sc Proof:}}{~\hfill QED}
% \newenvironment{AMS}{}{}
% \newenvironment{keywords}{}{}
%%%%%%%%%%%%%%%%%%%%%%%%%%%%%%%%%%%%%%%%%%%%%%%%%%%%%%%
% %   The first thanks indicates your affiliation
% %
% %  Just the name here.
% %
% % Your mailing address goes at the end.
% %
% % \thanks is also how you indicate grant support
% %
%%%%%%%%%%%%%%%%%%%%%%%%%%%%%%%%%%%%%%%%%%%%%%%%%%%%%%%


\begin{document}
\setstretch{1.5} %줄간격 조정
\newpage
\section{코드 주석}
\subsection{\texttt{makeYX.m}}
  사실 수학적인 모델만 파악되면 코드가 하는 건 그다지 어렵지 않다. 왜냐하면 베이지언 모델은 사전분포와 가능도 함수만 구할 수 있으면 MCMC를 짜서 돌리면 사후분포로부터 샘플을 얻을 수 있기 때문이다. 물론 빠르게 target distribution로 수렴하는 MCMC를 고안하는 것은 여전히 연구대상이다.\par
  먼저 \texttt{makeYX.m}의 코드를 뜯어보면 다음의 모델을 고려했다는 것을 알 수 있다.
  \begin{align}
    y_{t}' &= y_{t-1}'\Phi_{1}'+\cdots +y_{t-p}'\Phi_{p}+e_{t}'\\
    &= \underbrace{\begin{bmatrix}y_{t-1}'& y_{t-2}' & \cdots & y_{t-p}' \end{bmatrix}}_{x_{t}'}\underbrace{\begin{bmatrix}\Phi_{1}'\\ \Phi_{2}'\\ \vdots \\ \Phi_{p}' \end{bmatrix}}_{\beta}+e_{t}'\\
    &= x_{t}'\beta + e_{t}',\quad e_{t}\sim \mathcal{N}_{k}\left(\mathbf{0},\Omega\right)
  \end{align}
  \begin{itemize}
    \item $\Phi_{t}:\; k\times k$
    \item $y_{t},e_{t}:\; k\times 1$
    \item $\beta: \; pk\times k$
    \item $x_{t}':\; 1\times pk$
  \end{itemize}
  그리고 모든 가능한 $t = (p+1),\ldots, T$에 해당하는 방정식들을 한번에 다음과 같이 나타낼 수 있다.
  \begin{align}
    \underbrace{\begin{bmatrix}y_{p+1}'\\ \vdots \\ y_{T}'  \end{bmatrix}}_{Y_{0}} = \underbrace{\begin{bmatrix}X_{p+1}'\\ \vdots \\ x_{t}' \end{bmatrix}}_{YL}\beta + \begin{bmatrix}e_{p+1}'\\ \vdots \\ e_{T}' \end{bmatrix}
  \end{align}
  코드에서도 이와 마찬가지로 \texttt{Y0}와 \texttt{YL}을 만들어낸다.\par
  위 모델을 열벡터로 다시 바꾸면 $y_{t}=\beta'x_{t}+e_{t}$인데 여기서 $\beta' x_{t}$는 어차피 $k\times 1$ 벡터이므로 $\ve (\beta' x_{t})$로 써도 무방하다. 그리고 다음 관계를 이용하려 한다.
  \begin{equation}
    \ve (A_{1}A_{2}A_{3}) = \left(A_{1}'\otimes A_{3}\right)\ve \left(A_{2}\right)
  \end{equation}
  그러므로 우리는 $\ve \left(I_{k}\beta' x_{t}\right)$에 대해서
  \begin{equation}
    \ve \left(I_{k}\beta' x_{t}\right) = \left(I_{k}\otimes x_{t}\right)\ve \left(\beta'\right)
  \end{equation}
  로 나타낼 수 있다. 코드에서는 모든 $t$에 대해 $I_{k}\otimes x_{t}$를 해서 3차원 array에 저장한 게 \texttt{YLm}이다. \par
\subsection{\texttt{Gen\_Phi.m, Gen\_Omega.m}}
  데이터셋이 만들어졌으니 이제 해야 하는 건 모수를 추정하는 일이다. 사실 업데이트하는 방법은 책에 나와있기는 하지만 가능도함수를 계산하는 것이 빠져있다. 원래대로 하자면
  \begin{equation}
    L\left(\left\{\Phi_{t}\right\}_{t=1}^{p},\Omega\,|\,D \right) = \prod_{t=1}^{p}p\left(y_{t}\right)\prod_{t=p+1}^{T}p\left(y_{t}\,;\,\beta' x_{t},\Omega\right)
  \end{equation}
  이지만 $1$부터 $p$까지의 개별적인 밀도함수는 알 수 없으므로 penalized likelihood를 쓴다.
  \begin{align}
    L\left(\left\{\Phi_{t}\right\}_{t=1}^{p},\Omega\,|\,D \right) &= \prod_{t=p+1}^{T}p\left(y_{t}\,;\, \beta'x_{t},\Omega\right)\\
    &= |\Omega|^{-(T-p-1)/2}\exp\left[-\dfrac{1}{2}\sum_{t=p+1}^{T}\left(y_{t}-\beta' x_{t}\right)'\Omega^{-1}\left(y_{t}-\beta' x_{t}\right) \right]
  \end{align}
  이제 이걸 바탕으로 추정을 하면 되는데 마지막으로 첨언을 하자면 코드에서 \texttt{ehat*ehat'} 이런 식으로 벡터의 텐서곱을 하는 부분이 있는데 이것은 저 가능도 함수에서
  \begin{equation}
    \left(y_{t}-\beta' x_{t}\right)'\Omega^{-1}\left(y_{t}-\beta' x_{t}\right) = \tr\left(\Omega\left(y_{t}-\beta' x_{t}\right)\left(y_{t}-\beta' x_{t}\right)'\right)
  \end{equation}
  이기 때문에 나온 부분이다. 나머지는 이전에 Gibbs sampler 구하듯이 똑같이 하면 된다.
\end{document}
