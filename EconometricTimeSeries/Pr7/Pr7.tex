\title{Economic Time Series HW2}
\author{Daeyoung Lim}

\documentclass[answers]{exam}
\usepackage[left=3cm,right=3cm,top=3.5cm,bottom=2cm]{geometry}
\usepackage{amssymb,amsmath,amsfonts,amsthm}
\usepackage{mathtools}
\usepackage{graphicx}
\usepackage{kotex}
\usepackage[utf8]{inputenc}
\usepackage[T1]{fontenc}
\usepackage{lmodern}
% \usepackage{enumerate}
\usepackage{listings}
\usepackage{courier}
\usepackage{cancel}
\usepackage{array}
\usepackage{courier}
\usepackage{booktabs}
\usepackage{titlesec}
\usepackage[shortlabels]{enumitem}
\usepackage{setspace}
\usepackage{newtxtext}
\usepackage[lite,nofontinfo,zswash,straightbraces]{mtpro2}
\usepackage{empheq}
\usepackage{tikz}
\usepackage{listings}
\usepackage{titlesec}

% \usepackage[toc,page]{appendix}

\setlength{\heavyrulewidth}{1.5pt}
\setlength{\abovetopsep}{4pt}

\DeclarePairedDelimiter{\ceil}{\lceil}{\rceil}
\newcommand\encircle[1]{%
  \tikz[baseline=(X.base)] 
    \node (X) [draw, shape=circle, inner sep=0] {\strut #1};}
 
% Command "alignedbox{}{}" for a box within an align environment
% Source: http://www.latex-community.org/forum/viewtopic.php?f=46&t=8144
\newlength\dlf  % Define a new measure, dlf
\newcommand\alignedbox[2]{
% Argument #1 = before & if there were no box (lhs)
% Argument #2 = after & if there were no box (rhs)
&  % Alignment sign of the line
{
\settowidth\dlf{$\displaystyle #1$}  
    % The width of \dlf is the width of the lhs, with a displaystyle font
\addtolength\dlf{\fboxsep+\fboxrule}  
    % Add to it the distance to the box, and the width of the line of the box     ㅊ
\hspace{-\dlf}  
    % Move everything dlf units to the left, so that & #1 #2 is aligned under #1 & #2
\boxed{#1 #2}
    % Put a box around lhs and rhs
}
}
\setcounter{secnumdepth}{4}
\lstset{
         basicstyle=\footnotesize\ttfamily, % Standardschrift
         %numbers=left,               % Ort der Zeilennummern
         numberstyle=\tiny,          % Stil der Zeilennummern
         %stepnumber=2,               % Abstand zwischen den Zeilennummern
         numbersep=5pt,              % Abstand der Nummern zum Text
         tabsize=2,                  % Groesse von Tabs
         extendedchars=true,         %
         breaklines=true,            % Zeilen werden Umgebrochen
         keywordstyle=\color{red},
            frame=b,         
 %        keywordstyle=[1]\textbf,    % Stil der Keywords
 %        keywordstyle=[2]\textbf,    %
 %        keywordstyle=[3]\textbf,    %
 %        keywordstyle=[4]\textbf,   \sqrt{\sqrt{}} %
         stringstyle=\color{white}\ttfamily, % Farbe der String
         showspaces=false,           % Leerzeichen anzeigen ?
         showtabs=false,             % Tabs anzeigen ?
         xleftmargin=17pt,
         framexleftmargin=17pt,
         framexrightmargin=5pt,
         framexbottommargin=4pt,
         %backgroundcolor=\color{lightgray},
         showstringspaces=false      % Leerzeichen in Strings anzeigen ?        
 }
 \lstloadlanguages{% Check Dokumentation for further languages ...
         %[Visual]Basic
         %Pascal
         %C
         %C++
         %XML
         %HTML
         Java
 }
    %\DeclareCaptionFont{blue}{\color{blue}} 

\definecolor{myblue}{RGB}{72, 165, 226}
\definecolor{myorange}{RGB}{222, 141, 8}
\titleformat{\paragraph}
{\normalfont\normalsize\bfseries}{\theparagraph}{1em}{}
\titlespacing*{\paragraph}
{0pt}{3.25ex plus 1ex minus .2ex}{1.5ex plus .2ex}
\setlength{\heavyrulewidth}{1.5pt}
\setlength{\abovetopsep}{4pt}
\setlength{\parindent}{0mm}
\linespread{1.3}
\DeclareMathOperator{\sech}{sech}
\DeclareMathOperator{\csch}{csch}
\DeclareMathOperator*{\argmin}{\arg\!\min}
\DeclareMathOperator{\Tr}{Tr}

\newcommand{\bs}{\boldsymbol}
\newcommand{\opn}{\operatorname}
%%%%%%%%%%%%%%%%%%%%%%%%%%%%%%%%%%%%%%%%%%%%%%%%%%%%%%%
% % We use newtheorem to define theorem-like structures
% %
% % Here are some common ones. . .
%%%%%%%%%%%%%%%%%%%%%%%%%%%%%%%%%%%%%%%%%%%%%%%%%%%%%%%
% \newtheorem{theorem}{Theorem}
% \newtheorem{lemma}{Lemma}
% \newtheorem{proposition}{Proposition}
% \newtheorem{scolium}{Scolium}   %% And a not so common one.
% \newtheorem{definition}{Definition}
% \newenvironment{proof}{{\sc Proof:}}{~\hfill QED}
% \newenvironment{AMS}{}{}
% \newenvironment{keywords}{}{}
%%%%%%%%%%%%%%%%%%%%%%%%%%%%%%%%%%%%%%%%%%%%%%%%%%%%%%%
% %   The first thanks indicates your affiliation
% %
% %  Just the name here.
% %
% % Your mailing address goes at the end.
% %
% % \thanks is also how you indicate grant support
% %
%%%%%%%%%%%%%%%%%%%%%%%%%%%%%%%%%%%%%%%%%%%%%%%%%%%%%%%


\begin{document}
\setstretch{1.5} %줄간격 조정
\newpage
\firstpageheader{}{}{\bf\large Daeyoung Lim \\ Economic Time Series \\ Fall Semester, 2016}
\runningheader{Daeyoung Lim}{Economic Time Series}{Fall Semester, 2016}

\begin{questions}
    \question
    Consider the following MA(2) process:
    \begin{equation}
      x_{t} = (1+2.4L+0.8L^{2})e_{t}
    \end{equation}
    where $e_{t} \sim \mathrm{WN}(0,1)$.
    \begin{enumerate}[(a)]
      \item Is this process stationary? Prove it using autocovariance function.
      \item Is it invertible? If so, prove. Otherwise, find the invertible representation for the process. Calculate the autocovariance of the invertible representation and verify that they are the same as those obtained in (a).
    \end{enumerate}
    \begin{solution}
      \begin{enumerate}[(a)]
        \item Since $e_{t}$ is a white noise, its mean is zero and its autocovariance function is
        \begin{equation}
          R_{e}(h) = \begin{cases}1, & \text{if $h=0$}\\ 0, & \text{otherwise} \end{cases}=\mathbb{I}_{\left\{0\right\}}(h)
        \end{equation}
        Moreover, if we define $\psi(L)=1+2.4L+0.8L^{2}$, it follows that
        \begin{equation}
          \sum_{j=-\infty}^{\infty}\left|\psi_{j}^{2}\right|<\infty.
        \end{equation}
        By proposition 1 in p.50 in the lecture note, $x_{t}$ is also stationary whose covariance function is
        \begin{align}
          R_{x}(h) &= \sum_{j,k=-\infty}^{\infty}\psi_{j}\psi_{k}R_{e}(h-j+k)\\
          &= \sum_{j,k=1}^{2}\psi_{j}\psi_{k}\mathbb{I}_{\left\{0\right\}}(h-j+k)\\
          &= \begin{cases}1+\psi_{1}^{2}+\psi_{2}^{2},& \text{if $h=0$}\\\psi_{1}^{2}+\psi_{1}\psi_{2}, & \text{if $|h|=1$}\\ \psi_{2}, & \text{if $|h|=2$}\\ 0, & \text{if $|h|>2$} \end{cases}
        \end{align}
        \item By theorem 2 in p.52 of the lecture note, $A(L)=1$ and $\Psi(L)=1+\psi_{1}L+\psi_{2}L^{2}$ have no common zeros since
        \begin{equation}
          \Psi(z) = 1+2.4z+0.8z^{2}=0
        \end{equation}
        to which the solutions are $z=-2.5$ and $z=-0.5$. Therefore, the process is non-invertible. To get the invertible representation, we should reciprocate the problematic root, which yields
        \begin{equation}
          \tilde{x}_{t} = \left(1-\tilde{\psi}_{1}L-\tilde{\psi}_{2}L^{2}\right)\tilde{e}_{t}, \quad \tilde{e}_{t}\sim\mathrm{WN}\left(0,\sigma^{2}\right)
        \end{equation}
        where $\tilde{\psi}_{1}=-4.5$, $\tilde{\psi}_{2}=-5$, and $\sigma^{2}=0.16$. This comes from the factorized equation
        \begin{equation}
          \left(1+2z\right)\left(1+\dfrac{z}{0.4}\right)
        \end{equation}
        so as to push the roots outside of the unit circle.
      For any general MA(2) model, the autocovariance function becomes as follows,
      \begin{align}
        R_{\tilde{x}}(h) = \begin{cases}\sigma^{2}\left(1+\tilde{\psi}_{1}^{2}+\tilde{\psi}_{2}^{2}\right)=7.4, & \text{if $h=0$}\\ -\sigma^{2}\tilde{\psi}_{1}\left(1-\tilde{\psi}_{2}\right)=4.32,& \text{if $|h|=1$}\\ -\sigma^{2}\tilde{\psi}_{2}=0.8, & \text{if $|h|=2$}\\ 0, & \text{if $|h|>2$} \end{cases}
      \end{align}
      Thus, $\tilde{x}_{t}=\left(1+4.5L+5L^{2}\right)\tilde{e}_{t}$ becomes the invertible counterpart.
      \end{enumerate}
    \end{solution}
\end{questions}
\end{document}
