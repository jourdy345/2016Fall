\title{Stochastic Dynamic Programming HW2}
\author{Daeyoung Lim}

\documentclass[answers]{exam}
\usepackage[left=3cm,right=3cm,top=3.5cm,bottom=2cm]{geometry}
\usepackage{amssymb,amsmath,amsfonts,amsthm}
\usepackage{mathtools}
\usepackage{graphicx}
\usepackage{kotex}
\usepackage[utf8]{inputenc}
\usepackage[T1]{fontenc}
\usepackage{lmodern}
% \usepackage{enumerate}
\usepackage{listings}
\usepackage{courier}
\usepackage{cancel}
\usepackage{array}
\usepackage{courier}
\usepackage{booktabs}
\usepackage{titlesec}
\usepackage[shortlabels]{enumitem}
\usepackage{setspace}
\usepackage{newtxtext}
\usepackage[lite,nofontinfo,zswash,straightbraces]{mtpro2}
\usepackage{empheq}
\usepackage{tikz}
\usepackage{listings}

% \usepackage[toc,page]{appendix}

\setlength{\heavyrulewidth}{1.5pt}
\setlength{\abovetopsep}{4pt}

\DeclarePairedDelimiter{\ceil}{\lceil}{\rceil}
\newcommand\encircle[1]{%
  \tikz[baseline=(X.base)] 
    \node (X) [draw, shape=circle, inner sep=0] {\strut #1};}
 
% Command "alignedbox{}{}" for a box within an align environment
% Source: http://www.latex-community.org/forum/viewtopic.php?f=46&t=8144
\newlength\dlf  % Define a new measure, dlf
\newcommand\alignedbox[2]{
% Argument #1 = before & if there were no box (lhs)
% Argument #2 = after & if there were no box (rhs)
&  % Alignment sign of the line
{
\settowidth\dlf{$\displaystyle #1$}  
    % The width of \dlf is the width of the lhs, with a displaystyle font
\addtolength\dlf{\fboxsep+\fboxrule}  
    % Add to it the distance to the box, and the width of the line of the box     ㅊ
\hspace{-\dlf}  
    % Move everything dlf units to the left, so that & #1 #2 is aligned under #1 & #2
\boxed{#1 #2}
    % Put a box around lhs and rhs
}
}
\setcounter{secnumdepth}{4}
\lstset{
         basicstyle=\footnotesize\ttfamily, % Standardschrift
         %numbers=left,               % Ort der Zeilennummern
         numberstyle=\tiny,          % Stil der Zeilennummern
         %stepnumber=2,               % Abstand zwischen den Zeilennummern
         numbersep=5pt,              % Abstand der Nummern zum Text
         tabsize=2,                  % Groesse von Tabs
         extendedchars=true,         %
         breaklines=true,            % Zeilen werden Umgebrochen
         keywordstyle=\color{red},
            frame=b,         
 %        keywordstyle=[1]\textbf,    % Stil der Keywords
 %        keywordstyle=[2]\textbf,    %
 %        keywordstyle=[3]\textbf,    %
 %        keywordstyle=[4]\textbf,   \sqrt{\sqrt{}} %
         stringstyle=\color{white}\ttfamily, % Farbe der String
         showspaces=false,           % Leerzeichen anzeigen ?
         showtabs=false,             % Tabs anzeigen ?
         xleftmargin=17pt,
         framexleftmargin=17pt,
         framexrightmargin=5pt,
         framexbottommargin=4pt,
         %backgroundcolor=\color{lightgray},
         showstringspaces=false      % Leerzeichen in Strings anzeigen ?        
 }
 \lstloadlanguages{% Check Dokumentation for further languages ...
         %[Visual]Basic
         %Pascal
         %C
         %C++
         %XML
         %HTML
         Java
 }
    %\DeclareCaptionFont{blue}{\color{blue}} 

\definecolor{myblue}{RGB}{72, 165, 226}
\definecolor{myorange}{RGB}{222, 141, 8}
\titleformat{\paragraph}
{\normalfont\normalsize\bfseries}{\theparagraph}{1em}{}
\titlespacing*{\paragraph}
{0pt}{3.25ex plus 1ex minus .2ex}{1.5ex plus .2ex}
\setlength{\heavyrulewidth}{1.5pt}
\setlength{\abovetopsep}{4pt}
\setlength{\parindent}{0mm}
\linespread{1.3}
\DeclareMathOperator{\sech}{sech}
\DeclareMathOperator{\csch}{csch}
\DeclareMathOperator*{\argmin}{\arg\!\min}
\DeclareMathOperator{\Tr}{Tr}
\DeclareMathOperator{\Prob}{P}

\newcommand{\bs}{\boldsymbol}
\newcommand{\opn}{\operatorname}
%%%%%%%%%%%%%%%%%%%%%%%%%%%%%%%%%%%%%%%%%%%%%%%%%%%%%%%
% % We use newtheorem to define theorem-like structures
% %
% % Here are some common ones. . .
%%%%%%%%%%%%%%%%%%%%%%%%%%%%%%%%%%%%%%%%%%%%%%%%%%%%%%%
% \newtheorem{theorem}{Theorem}
% \newtheorem{lemma}{Lemma}
% \newtheorem{proposition}{Proposition}
% \newtheorem{scolium}{Scolium}   %% And a not so common one.
% \newtheorem{definition}{Definition}
% \newenvironment{proof}{{\sc Proof:}}{~\hfill QED}
% \newenvironment{AMS}{}{}
% \newenvironment{keywords}{}{}
%%%%%%%%%%%%%%%%%%%%%%%%%%%%%%%%%%%%%%%%%%%%%%%%%%%%%%%
% %   The first thanks indicates your affiliation
% %
% %  Just the name here.
% %
% % Your mailing address goes at the end.
% %
% % \thanks is also how you indicate grant support
% %
%%%%%%%%%%%%%%%%%%%%%%%%%%%%%%%%%%%%%%%%%%%%%%%%%%%%%%%


\begin{document}
\setstretch{1.5} %줄간격 조정
\newpage
\firstpageheader{}{}{\bf\large Daeyoung Lim \\ Stochastic Dynamic Programming \\ Fall Semester, 2016}
\runningheader{Daeyoung Lim}{Stochastic Dynamic Programming}{Fall Semester, 2016}
\begin{questions}
  \question
  Suppose that a store has two of these machines. Assume that the machines fail and are repaired independently of each other. Let $Y_{n}$ be the number of machines that are working on day $n$. Is $\left\{Y_{n}\,:\, n\geq 1 \right\}$ a Markov chain?
  \begin{solution}
    Recall $X_{n}$ from the previous problem where $X_{n}$ is the state of a machine that is set to $0$ if it is not working and $1$ if it is on day $n$. Since the two machines are repaired independently, we have two such $i.i.d.$ random variables $X_{n}^{m},\; m=1,2$ where $m$ indicates the machine number. The row-stochastic transition matrix of $X_{n}^{m}$ is as follows:
    \begin{equation}
        T_{X} = \begin{bmatrix} 0.3  & 0.7\\ 0.4 & 0.6 \end{bmatrix}
    \end{equation}
    Now, let's turn to $Y_{n}$ which denotes the number of machines functioning on day $n$. The state space is obviously $S = \left\{0,1,2\right\}$ for we have two machines. Then, if we let $\wedge$ denote logical conjuction (\texttt{and}) and $\vee$ logical disjuction (\texttt{or}),
    \begin{align}
        \Prob(Y_{n}=2\,|\,Y_{n-1}=2) &= \Prob((1\to 1)\wedge (1\to1)) = 0.36\\
        \Prob(Y_{n}=1\,|\,Y_{n-1}=2) &= \binom{2}{1}\Prob((1\to1)\wedge (1\to0)) = 2\times 0.6\times 0.4 = 0.48\\
        \Prob(Y_{n}=0\,|\,Y_{n-1}=2) &= \Prob((1\to0)\wedge (1\to0)) = 0.16\\
        \Prob(Y_{n}=2\,|\, Y_{n-1}=1) &= \Prob((1\to1)\wedge(0\to1)) = 0.6\times0.7 = 0.42\\
        \Prob(Y_{n}=1\,|\,Y_{n-1}=1) &= \Prob((1\to1)\wedge (0\to 0))+\Prob((1\to0)\wedge (0\to1)) = 0.18+0.28 = 0.46\\
        \Prob(Y_{n}=0\,|\,Y_{n-1}=1) &= \Prob((1\to0)\wedge(0\to0)) = 0.4\times 0.3 = 0.12\\
        \Prob(Y_{n}=2\,|\,Y_{n-1}=0) &= \Prob((0\to1)\wedge(0\to1)) = 0.7^{2} = 0.49\\
        \Prob(Y_{n}=1\,|\,Y_{n-1}=0) &= \binom{2}{1}\Prob((0\to1)\wedge(0\to0)) = 2\times0.7\times0.3 = 0.42\\
        \Prob(Y_{n}=0\,|\,Y_{n-1}=0) &= \Prob((0\to0)\wedge(0\to0)) = 0.09
    \end{align}
    where $i\to j$ indicates the event of an arbitrary machine state transitioning from $i$ to $j$, i.e., $X_{n}=j\,|\,X_{n-1}=i$.
    Thus, the one-step transition matrix is
    \begin{equation}
        T_{Y} = \begin{bmatrix} 0.09 & 0.42 & 0.49 \\ 0.12 & 0.46 & 0.42 \\ 0.16 & 0.48 & 0.36 \end{bmatrix}
    \end{equation}
    Every row of $T_{Y}$ adds up to unity proving it a legitimate transition matrix. Thus, $\left\{Y_{n}\,:\,n\geq 1\right\}$ is a Markov chain.
  \end{solution}
\end{questions}
\end{document}
