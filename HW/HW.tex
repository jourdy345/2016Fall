\title{Statistics for Economics}
\author{Daeyoung Lim}

\documentclass[answers]{exam}
\usepackage[left=3cm,right=3cm,top=3.5cm,bottom=2cm]{geometry}
\usepackage{amssymb,amsmath,amsfonts,amsthm}
\usepackage{mathtools}
\usepackage{graphicx}
\usepackage{kotex}
\usepackage[utf8]{inputenc}
\usepackage[T1]{fontenc}
\usepackage{lmodern}
% \usepackage{enumerate}
\usepackage{listings}
\usepackage{courier}
\usepackage{cancel}
\usepackage{array}
\usepackage{courier}
\usepackage{booktabs}
\usepackage{titlesec}
\usepackage[shortlabels]{enumitem}
\usepackage{setspace}
\usepackage{newtxtext}
\usepackage[lite,nofontinfo,zswash,straightbraces]{mtpro2}
\usepackage{empheq}
\usepackage{tikz}
\usepackage{listings}

% \usepackage[toc,page]{appendix}

\setlength{\heavyrulewidth}{1.5pt}
\setlength{\abovetopsep}{4pt}

\DeclarePairedDelimiter{\ceil}{\lceil}{\rceil}
\newcommand\encircle[1]{%
  \tikz[baseline=(X.base)] 
    \node (X) [draw, shape=circle, inner sep=0] {\strut #1};}
 
% Command "alignedbox{}{}" for a box within an align environment
% Source: http://www.latex-community.org/forum/viewtopic.php?f=46&t=8144
\newlength\dlf  % Define a new measure, dlf
\newcommand\alignedbox[2]{
% Argument #1 = before & if there were no box (lhs)
% Argument #2 = after & if there were no box (rhs)
&  % Alignment sign of the line
{
\settowidth\dlf{$\displaystyle #1$}  
    % The width of \dlf is the width of the lhs, with a displaystyle font
\addtolength\dlf{\fboxsep+\fboxrule}  
    % Add to it the distance to the box, and the width of the line of the box     ㅊ
\hspace{-\dlf}  
    % Move everything dlf units to the left, so that & #1 #2 is aligned under #1 & #2
\boxed{#1 #2}
    % Put a box around lhs and rhs
}
}
\setcounter{secnumdepth}{4}
\lstset{
         basicstyle=\footnotesize\ttfamily, % Standardschrift
         %numbers=left,               % Ort der Zeilennummern
         numberstyle=\tiny,          % Stil der Zeilennummern
         %stepnumber=2,               % Abstand zwischen den Zeilennummern
         numbersep=5pt,              % Abstand der Nummern zum Text
         tabsize=2,                  % Groesse von Tabs
         extendedchars=true,         %
         breaklines=true,            % Zeilen werden Umgebrochen
         keywordstyle=\color{red},
            frame=b,         
 %        keywordstyle=[1]\textbf,    % Stil der Keywords
 %        keywordstyle=[2]\textbf,    %
 %        keywordstyle=[3]\textbf,    %
 %        keywordstyle=[4]\textbf,   \sqrt{\sqrt{}} %
         stringstyle=\color{white}\ttfamily, % Farbe der String
         showspaces=false,           % Leerzeichen anzeigen ?
         showtabs=false,             % Tabs anzeigen ?
         xleftmargin=17pt,
         framexleftmargin=17pt,
         framexrightmargin=5pt,
         framexbottommargin=4pt,
         %backgroundcolor=\color{lightgray},
         showstringspaces=false      % Leerzeichen in Strings anzeigen ?        
 }
 \lstloadlanguages{% Check Dokumentation for further languages ...
         %[Visual]Basic
         %Pascal
         %C
         %C++
         %XML
         %HTML
         Java
 }
    %\DeclareCaptionFont{blue}{\color{blue}} 

\definecolor{myblue}{RGB}{72, 165, 226}
\definecolor{myorange}{RGB}{222, 141, 8}
\titleformat{\paragraph}
{\normalfont\normalsize\bfseries}{\theparagraph}{1em}{}
\titlespacing*{\paragraph}
{0pt}{3.25ex plus 1ex minus .2ex}{1.5ex plus .2ex}
\setlength{\heavyrulewidth}{1.5pt}
\setlength{\abovetopsep}{4pt}
\setlength{\parindent}{0mm}
\linespread{1.3}
\DeclareMathOperator{\sech}{sech}
\DeclareMathOperator{\csch}{csch}
\DeclareMathOperator*{\argmin}{\arg\!\min}
\DeclareMathOperator{\Tr}{Tr}

\newcommand{\bs}{\boldsymbol}
\newcommand{\opn}{\operatorname}
%%%%%%%%%%%%%%%%%%%%%%%%%%%%%%%%%%%%%%%%%%%%%%%%%%%%%%%
% % We use newtheorem to define theorem-like structures
% %
% % Here are some common ones. . .
%%%%%%%%%%%%%%%%%%%%%%%%%%%%%%%%%%%%%%%%%%%%%%%%%%%%%%%
% \newtheorem{theorem}{Theorem}
% \newtheorem{lemma}{Lemma}
% \newtheorem{proposition}{Proposition}
% \newtheorem{scolium}{Scolium}   %% And a not so common one.
% \newtheorem{definition}{Definition}
% \newenvironment{proof}{{\sc Proof:}}{~\hfill QED}
% \newenvironment{AMS}{}{}
% \newenvironment{keywords}{}{}
%%%%%%%%%%%%%%%%%%%%%%%%%%%%%%%%%%%%%%%%%%%%%%%%%%%%%%%
% %   The first thanks indicates your affiliation
% %
% %  Just the name here.
% %
% % Your mailing address goes at the end.
% %
% % \thanks is also how you indicate grant support
% %
%%%%%%%%%%%%%%%%%%%%%%%%%%%%%%%%%%%%%%%%%%%%%%%%%%%%%%%


\begin{document}
\setstretch{1.5} %줄간격 조정
\newpage
\firstpageheader{}{}{\bf\large Daeyoung Lim \\ Statistics for Economics \\ Fall, 2016}
\runningheader{Daeyoung Lim}{Statistics for Economics}{Fall, 2016}
\begin{questions}
  \question
  The sample space $\Omega$ is $\Omega=[0,1]$ and the probability can be expressed as follows.
  \begin{equation}
    p(x) = 2xI\left\lbrace 0\leq x \leq 1\right\rbrace
  \end{equation}
  Let random variables $X$ and $Y$ be defined as below.
  \begin{align}
    X(\omega) = \begin{cases}1, & \text{if $0\leq \omega < 1/4$}\\ 0, & \text{if $1/4\leq \omega < 1/2$}\\ -1, & \text{if $1/2\leq \omega <3/4$}\\ 0, & \text{if $3/4 \leq \omega \leq 1$} \end{cases}, \qquad Y(\omega) = \begin{cases}1, & \text{if $0\leq \omega < 1/2$}\\ 0, & \text{if $1/2\leq \omega \leq 1$} \end{cases}
  \end{align}
  \begin{enumerate}[(1)]
    \item Compute the joint density of $X$ and $Y$.
    \item Compute the marginal densities of $X$ and $Y$ respectively.
    \item Compute the expectations of $X$ and $Y$ respectively.
    \item Compute the conditional expectation $\mathbf{E}(X\,|\,Y)$.
  \end{enumerate}
  \begin{solution}
    \begin{enumerate}[(1)]
      \item If we get every combination of possible values of $X$ and $Y$, then we can say that we have obtained the joint probability mass function of $X$ and $Y$.
      \begin{itemize}
        \item $\Pr(X=0,Y=0)$
        \begin{align}
          &= \Pr\left[\left(\left\lbrace \omega\,|\, \dfrac{1}{4}\leq\omega <\dfrac{1}{2}\right\rbrace \cup \left\lbrace \omega\,|\, \dfrac{3}{4}\leq \omega \leq 1 \right\rbrace\right)\cap \left\lbrace \omega\,|\, \dfrac{1}{2}\leq \omega\leq 1\right\rbrace\right]\\
          &= \Pr\left[\left\lbrace \omega\,|\, \dfrac{3}{4}\leq \omega\leq 1\right\rbrace \right]\\
          &= \int\limits_{3/4}^{1}2x\,dx = 1-\dfrac{9}{16}\\
          &= \dfrac{7}{16}
        \end{align}
        \item $\Pr(X=0,Y=1)$
        \begin{align}
          &= \Pr\left[\left(\left\lbrace \omega\,|\, \dfrac{1}{4}\leq\omega <\dfrac{1}{2}\right\rbrace \cup \left\lbrace \omega\,|\, \dfrac{3}{4}\leq \omega \leq 1 \right\rbrace\right)\cap \left\lbrace \omega\,|\, 0\leq \omega< \dfrac{1}{2}\right\rbrace\right]\\
          &= \Pr\left[\left\lbrace \omega\,|\, \dfrac{1}{4}\leq \omega< \dfrac{1}{2}\right\rbrace \right]\\
          &= \int\limits_{1/4}^{1/2}2x\,dx= \dfrac{1}{4}-\dfrac{1}{16}\\
          &= \dfrac{3}{16}
        \end{align}
        \item $\Pr(X=1,Y=0)$
        \begin{align}
          &= \Pr\left[\left\lbrace \omega\,|\,0\leq \omega < \dfrac{1}{4}\right\rbrace \cup \left\lbrace \omega\,|\, \dfrac{1}{2}\leq \omega\leq 1\right\rbrace \right]\\
          &= \Pr\left[\emptyset\right]\\
          &= 0
        \end{align}
        \item $\Pr(X=1,Y=1)$
        \begin{align}
          &= \Pr\left[\left\lbrace \omega\,|\, 0\leq \omega<\dfrac{1}{4}\right\rbrace \cup \left\lbrace 0\leq \omega < \dfrac{1}{2}\right\rbrace \right]\\
          &= \Pr\left[\left\lbrace \omega\,|\,0\leq \omega < \dfrac{1}{2}\right\rbrace\right]\\
          &= \int\limits_{0}^{1/4}2x\,dx = \dfrac{1}{16}
        \end{align}
        \item $\Pr(X=-1,Y=0)$
        \begin{align}
          &= \Pr\left[\left\lbrace \omega\,|\, \dfrac{1}{2}\leq \omega < \dfrac{3}{4}\right\rbrace \cup \left\lbrace \omega\,|\, \dfrac{1}{2}\leq \omega \leq 1 \right\rbrace \right]\\
          &= \Pr\left[\left\lbrace \omega\,|\, \dfrac{1}{2}\leq \omega < \dfrac{3}{4}\right\rbrace \right]\\
          &= \int\limits_{1/2}^{3/4}2x\,dx = \dfrac{9}{16}-\dfrac{1}{4}\\
          &= \dfrac{5}{16}
        \end{align}
        \item $\Pr(X=-1,Y=1)$
        \begin{align}
          &= \Pr\left[\left\lbrace \omega\,|\, \dfrac{1}{2}\leq \omega < \dfrac{3}{4}\right\rbrace \cup \left\lbrace \omega\,|\,0\leq \omega \leq \dfrac{1}{2} \right\rbrace \right]\\
          &= \Pr\left(\left\lbrace\dfrac{1}{2} \right\rbrace \right)\\
          &= 0
        \end{align}
        because the probability for a singleton with respect to a continuous random variable is always zero.
      \end{itemize}
      \item For $X$,
      \begin{itemize}
        \item $\Pr(X=1)$
        \begin{align}
          &= \Pr\left[\left\lbrace \omega\,|\, 0\leq \omega < \dfrac{1}{4}\right\rbrace \right]\\
          &= \int\limits_{0}^{1/4}2x\,dx = \dfrac{1}{16}
        \end{align}
        \item $\Pr(X=0)$
        \begin{align}
          &= \Pr\left[\left\lbrace \omega\,|\,\dfrac{1}{4}\leq \omega < \dfrac{1}{2}\right\rbrace \cup \left\lbrace \omega\,|\, \dfrac{3}{4}\leq \omega\leq 1\right\rbrace \right]\\
          &= \int\limits_{1/4}^{1/4}2x\,dx + \int\limits_{3/4}^{1}2x\,dx= \left( \dfrac{1}{4}-\dfrac{1}{16}\right)+\left(1-\dfrac{9}{16}\right) \\
          &= \dfrac{5}{8}
        \end{align}
        \item $\Pr(X=-1)$
        \begin{align}
          &= \Pr\left[\left\lbrace \omega\,|\,\dfrac{1}{2}\leq \omega < \dfrac{3}{4}\right\rbrace \right]\\
          &= \int\limits_{1/2}^{3/4}2x\,dx = \dfrac{9}{16}-\dfrac{1}{4}\\
          &= \dfrac{5}{16}
        \end{align}
      \end{itemize}
      For $Y$,
      \begin{itemize}
        \item $\Pr(Y=0)$
        \begin{align}
          &= \Pr\left[\left\lbrace \omega\,|\, \dfrac{1}{2}\leq \omega\leq 1\right\rbrace \right]\\
          &= \int\limits_{1/2}^{1}2x\,dx = 1-\dfrac{1}{4}\\
          &= \dfrac{3}{4}
        \end{align}
        \item $\Pr(Y=1)$
        \begin{align}
          &= \Pr\left[\left\lbrace \omega\,|\,0\leq \omega < \dfrac{1}{2}\right\rbrace  \right]\\
          &= \int\limits_{0}^{1/2}2x\,dx\\
          &= \dfrac{1}{4}
        \end{align}
      \end{itemize}
      \item Since we know the marginal distributions of $X$ and $Y$, we in full position to compute the expectations.
      \begin{itemize}
        \item For $\mathbf{E}(X)$,
        \begin{align}
          \mathbf{E}(X) &= 1\times \dfrac{1}{16}+0\times \dfrac{5}{8}-1\times \dfrac{5}{16}\\
          &= -\dfrac{1}{4}
        \end{align}
        \item For $\mathbf{E}(Y)$,
        \begin{align}
          \mathbf{E}(Y) &= 0\times \dfrac{3}{4}+1\times \dfrac{1}{4}\\
          &= \dfrac{1}{4}
        \end{align}
      \end{itemize}
      \item By definition of the conditional expectation,
      \begin{align}
        \mathbf{E}(X\,|\,Y) &= \sum_{x\in \mathcal{X}}x\Pr(X=x\,|\,Y=y)
      \end{align}
      which means we have 2 different cases of $Y=y$ to be conditioned on.
      \begin{itemize}
          \item $\mathbf{E}(X\,|\,Y=1)$
          \begin{align}
            &= \sum_{x\in \mathcal{X}} x\Pr(X=x\,|\,Y=1)\\
            &= \sum_{x\in \mathcal{X}} x\dfrac{\Pr(X=x,Y=1)}{\Pr(Y=1)}\\
            &= 1\times \dfrac{\Pr(X=1,Y=1)}{\Pr(Y=1)}+0\times\dfrac{\Pr(X=0,Y=1)}{\Pr(Y=1)}-1\times\dfrac{\Pr(X=-1,Y=1)}{\Pr(Y=1)}
          \end{align}
          \item $\mathbf{E}(X\,|\,Y=0)$
          \begin{align}
            &= \sum_{x\in\mathcal{X}}x\Pr(X=x\,|\,Y=0)\\
            &= \sum_{x\in \mathcal{X}}x\dfrac{\Pr(X=x,Y=0)}{\Pr(Y=0)}\\
            &= 1\times \dfrac{\Pr(X=1,Y=0)}{\Pr(Y=0)}+0\times\dfrac{\Pr(X=0,Y=0)}{\Pr(Y=0)}-1\times\dfrac{\Pr(X=-1,Y=0)}{\Pr(Y=0)}
          \end{align}
      \end{itemize}
    \end{enumerate}
  \end{solution}
\end{questions}
\end{document}
